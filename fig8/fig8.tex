\documentclass[border=2pt]{standalone}
\usepackage[dvipsnames, svgnames, x11names]{xcolor}
\usepackage{tikz}
\usepackage{pgfplots}
\pgfplotsset{compat=1.5}
\pgfplotsset{width=10cm,compat=1.16}
\newcommand{\fz}[1]{\fontsize{#1 pt}{0.001mm}\selectfont}
\usepackage{pgfkeys}

\usetikzlibrary{calc,arrows.meta,positioning,intersections}
\begin{document}
	\begin{tikzpicture}
		
		\begin{axis}[smooth,
			axis line shift=6.5pt, grid=major,grid style={gray!24,opacity = 0.4,line width=0.02pt},
		title style ={ align = left ,font = \fz{10pt}},
		title={$p=10, c=4, s=3,w=6, B=100, r=0.05$\\$ \mu=100, \sigma=100$},
			xlabel={\fz{10}Acceptable Bankruptcy Risk: $\large \alpha$},
		ylabel={\fz{10}Channel Profit: $\large \tilde{\pi}(\tilde{q}^c)$},
%		grid=major,grid style={color=gray,opacity=0.23,line width =0.2pt},
		xtick={0,0.125,0.206,0.5,1},
		xmin =0,xmax=1,
		ymin =-100,ymax=703.05,
		ytick={-100,0,465 ,703.05},
		legend style={
			at={(0.75cm,64.8054cm)},
			anchor=west,cells={anchor=west},font=\footnotesize,
		}
		]		

		\legend{,$\tilde{\pi}(\tilde{q}^c) =q(p-c)-(p-s) \int_{0}^{q} F(x) d x-(c q-B)^+ r$};
			
		\fill [fill=MistyRose,opacity=0.38]  (0.12569,0) rectangle (0.21,465.22);
		
		\node at(0.1828,135) [rotate = 90,inner sep = 0pt,font=\fz{2pt},] {$\tilde{\pi}(\tilde{q}^c)$ increasing area};




		\addplot [gray!80!red,line width=0.47pt,,dash dot] table {piqc};
		\addplot [blue!80!red,line width=1pt] table {piqc_in};	
		% qs
		\addplot[blue!80!red,line width=1pt] coordinates {(0.205,465.218891871100) (1,465.218891871100)};

	
		
		
		\coordinate (c) at (0.206667,465.218891871100);	
		\draw[fill=DarkSalmon,draw = black!70] (c) circle (0.7mm) ;
		\node[align=left,anchor=west,font=\fz{7pt}] (n1) at ($(c)+(-10.mm,6.5mm)$) {
		$(\alpha,\tilde{q^c},\tilde{\pi},) = (0.21,194,465.22)$
		};
		\draw[-stealth,shorten > = 1mm,opacity=0.8] (n1)--(c);	
		
		

		
		
		
		\coordinate (c) at (0.12569,0);	
		\draw[fill=OrangeRed,draw = black!70] (c) circle (0.7mm) ;
		\node[align=left,anchor=west,font=\fz{7pt}] (n1) at ($(c)+(1.mm,-6.5mm)$) {
		$(\alpha,\tilde{q^c},\tilde{\pi},) = (0.13,0,0 )$
		};
		\draw[-stealth,shorten > = 1mm,opacity=0.8] (n1)--(c);	
		
		

	
			
	\end{axis}
\end{tikzpicture}
\end{document}